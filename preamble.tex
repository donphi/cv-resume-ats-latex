% ============================================================================
% PREAMBLE.TEX — Single Source of Truth
% ============================================================================
%
% This file controls the ENTIRE CV. Every visual decision lives here.
% Templates (header.tex, leftbox.tex, rightbox.tex) and components
% (summary.tex, research_experience.tex, etc.) consume ONLY named values
% defined in this file. No template contains a hardcoded color, font
% size, or spacing value.
%
% STRUCTURE:
%   §1  DOCUMENT CLASS & PACKAGES
%   §2  GRID ENGINE — Master dimensions everything derives from
%   §3  FONTS — Three families, all sizes centralised
%   §4  THEME — Colors in 3 tiers: Palette → Roles → Elements
%   §5  SPACING — Every vertical/horizontal gap as a named length
%   §6  BOX GEOMETRY — Widths, padding, column positions
%   §7  CONTENT FORMATTING — Lists, trees, timelines, helpers
%   §8  DEBUG
%
% ============================================================================


% ============================================================================
% §1  DOCUMENT CLASS & PACKAGES
% ============================================================================
%
% MASTER PARAMETERS (set before \documentclass — only \def allowed here):
%   \PageFormat .......... Paper size: "a4" (210×297mm) or "letter" (215.9×279.4mm)
%   \PageMarginMM ........ Page margin on all four sides, in mm (numeric)
% ============================================================================

% --- Page format (master) ---------------------------------------------------
%     Change to "letter" for US Letter, "a4" for ISO A4.
\def\PageFormat{a4}
\def\PageMarginMM{13.5}

% --- Page dimensions (derived — do not edit) --------------------------------
\def\tmpPageFmt{a4}
\ifx\PageFormat\tmpPageFmt
  \def\PageWidthMM{210}
  \def\PageHeightMM{297}
\else
  \def\PageWidthMM{215.9}
  \def\PageHeightMM{279.4}
\fi

\documentclass[\PageFormat paper]{article}

\usepackage{xfp}                  % floating-point math for grid calc
\usepackage{calc}                 % length arithmetic
\usepackage[absolute,overlay]{textpos}  % absolute positioning on page
\usepackage{fvextra}              % verbatim extras
\usepackage{xcolor}               % color system
\usepackage{fontspec}             % OpenType font loading (LuaLaTeX/XeLaTeX)
\usepackage{enumitem}             % list customisation
\usepackage{setspace}             % line spacing control
\usepackage{graphicx}             % images
\usepackage{hyperref}             % clickable links
\usepackage{fontawesome5}         % icons
\usepackage{pgfmath}              % math engine
\usepackage{pgffor}               % \foreach loops
\usepackage{xstring}              % string length measurement
\usepackage{tikz}                 % drawing (used sparingly)
\usepackage{geometry}             % page geometry (configured in §2)


% ============================================================================
% §2  GRID ENGINE
% ============================================================================
% The entire CV sits on a character-cell grid, like a terminal. Every
% position, width, height, and spacing is expressed in grid units.
%
% MASTER PARAMETERS (change these to resize the entire layout):
%   \GridFontSize .......... Base monospace font size in pt
%   \MonoWidthRatio ........ Width-to-height ratio of the monospace font
%                            (0.6 = Iosevka Extended / Grade 7)
%   \ContentWidthScale ..... Set to 1.0 (no scaling). Content minipage
%                            fills the full available space between padding.
%
% DERIVATION ORDER (margins are the hard authority):
%   1. \CellWidth, \CellHeight .... from font size, width ratio, and \PtToMM
%   2. \GridCols, \GridRows ........ whole cells that fit inside min margins
%   3. \DrawableWidth, \DrawableHeight .. grid * cell (exact, zero waste)
%   4. \MarginH, \MarginV ......... actual snapped margins (>= \PageMarginMM)
%   5. geometry + textpos ......... configured with snapped margins
%
%   This guarantees column 0 is exactly at the left margin and column
%   GridCols-1 is exactly at the right margin. Boxes pinned to either
%   edge will always touch their respective margin precisely.
%
% ALSO SET BY TEXTPOS (do not edit):
%   \TPHorizModule ......... Width of one grid column in pt
%   \TPVertModule .......... Height of one grid row in pt
% ============================================================================

% --- Master parameters ------------------------------------------------------
\newcommand{\GridFontSize}{9}             % pt — base size for mono grid
\newcommand{\MonoWidthRatio}{0.6}         % monospace width:height (0.6 = Extended)
\newcommand{\ContentWidthScale}{1.0}      % no scaling — minipage width = available space

% --- Step 1: cell size (from font metrics) ----------------------------------
%     1 pt = 25.4 mm / 72 ≈ 0.35278 mm  (standard TeX PostScript point)
%     We store this as \PtToMM so the conversion isn't a magic number.
\newcommand{\PtToMM}{\fpeval{25.4 / 72}}% = 0.35277… mm per pt
\newcommand{\CellWidth}{\fpeval{\GridFontSize * \MonoWidthRatio * \PtToMM}}
\newcommand{\CellHeight}{\fpeval{\GridFontSize * \PtToMM}}

% --- Step 2: grid dimensions (from minimum margin) --------------------------
%     How many whole cells fit between the margins?
\newcommand{\GridCols}{\fpeval{floor((\PageWidthMM - 2 * \PageMarginMM) / \CellWidth)}}
\newcommand{\GridRows}{\fpeval{floor((\PageHeightMM - 2 * \PageMarginMM) / \CellHeight)}}

% --- Step 3: exact drawable area (grid * cell = zero waste) -----------------
\newcommand{\DrawableWidth}{\fpeval{\GridCols * \CellWidth}}
\newcommand{\DrawableHeight}{\fpeval{\GridRows * \CellHeight}}

% --- Step 4: actual margins (snapped — always >= \PageMarginMM) -------------
%     The leftover space after fitting whole grid cells is split evenly
%     between both sides. This guarantees column 0 is exactly at the
%     left margin and column GridCols-1 is exactly at the right margin.
\newcommand{\MarginH}{\fpeval{(\PageWidthMM - \DrawableWidth) / 2}}
\newcommand{\MarginV}{\fpeval{(\PageHeightMM - \DrawableHeight) / 2}}

% --- Step 5: configure page geometry with snapped margins -------------------
\geometry{left=\MarginH mm, right=\MarginH mm, top=\MarginV mm, bottom=\MarginV mm}

% --- Activate grid ----------------------------------------------------------
%     We set \TPHorizModule and \TPVertModule directly instead of using
%     \TPGrid, because \TPGrid with the 'absolute' option divides
%     \paperwidth / cols (not drawable width), giving grid cells LARGER
%     than our monospace characters. Setting modules directly ensures
%     1 grid column = 1 character cell exactly.
\TPHorizModule=\fpeval{\CellWidth} mm\relax
\TPVertModule=\fpeval{\CellHeight} mm\relax
\textblockorigin{\MarginH mm}{\MarginV mm}% origin = top-left of snapped margins

% --- Kill all default LaTeX spacing -----------------------------------------
\setlength{\parindent}{0pt}
\setlength{\parskip}{0pt}
\setstretch{1.0}

% --- Disable hyphenation (prefer loose spacing over broken words) -----------
\hyphenpenalty=10000
\exhyphenpenalty=10000
\tolerance=9999
\emergencystretch=2em


% ============================================================================
% §3  FONTS
% ============================================================================
% Three font families, each with a specific role:
%
%   MONO (Iosevka Extended) — \mono
%     Purpose: Box borders, tree branches, all structural/decorative chars.
%     Fixed width means every character aligns to the grid perfectly.
%     Loaded at \GridFontSize so 1 character = 1 grid cell.
%
%   AILE (Iosevka Aile) — \aile
%     Purpose: Body text, descriptions, bullet content.
%     Proportional sans-serif. Readable at small sizes.
%     Content areas use \ContentWidthScale = 1.0 (no scaling).
%
%   ETOILE (Iosevka Etoile) — \etoile
%     Purpose: Available for display/heading use if needed.
%     Proportional serif variant. Currently unused but wired up.
%
% SIZE SCALE (all centralised here):
%   \ContentSizeLeft ..... Font size command for left box content
%   \ContentSizeRight .... Font size command for right box content
%   \HeadingSize ......... Font size for SubHead and Role titles
%   \SecondarySize ....... Font size for dates, locations, org lines
% ============================================================================

% --- Font loading -----------------------------------------------------------
\setmonofont{Iosevka-Extended}[
    Path=fonts/,
    Extension=.ttf,
    UprightFont=*,
    BoldFont=*Bold,
    ItalicFont=*Italic,
    BoldItalicFont=*Bold,
]

\setmainfont{IosevkaAile}[
    Path=fonts/,
    Extension=.ttf,
    UprightFont=*-Regular,
    BoldFont=*-Bold,
    ItalicFont=*-Italic,
    BoldItalicFont=*-BoldItalic,
]

\setsansfont{IosevkaEtoile}[
    Path=fonts/,
    Extension=.ttf,
    UprightFont=*-Regular,
    BoldFont=*-Bold,
    ItalicFont=*-Italic,
    BoldItalicFont=*-BoldItalic,
]

% --- Family shortcuts -------------------------------------------------------
\newcommand{\mono}{\ttfamily\fontsize{\GridFontSize pt}{\GridFontSize pt}\selectfont}
\newcommand{\aile}{\rmfamily}
\newcommand{\etoile}{\sffamily}

% --- Size commands (change these to resize all content globally) ------------
%     These are LaTeX size commands: \tiny \scriptsize \footnotesize
%     \small \normalsize \large etc.
\newcommand{\ContentSizeLeft}{\footnotesize}    % left box body text
\newcommand{\ContentSizeRight}{\footnotesize}   % right box body text (matches left for consistency)
\newcommand{\HeadingSize}{\small\bfseries}      % subsection headings
\newcommand{\SecondarySize}{\footnotesize}      % dates, orgs, locations


% ============================================================================
% §4  THEME — Color System
% ============================================================================
%
% 3 TIERS:
%
%   TIER 1: PALETTE
%     8 raw hex values. The ONLY place hex codes exist.
%     Swap one block to re-theme the entire CV.
%
%   TIER 2: ROLES
%     Maps palette → functional categories. Controls broad groups
%     (e.g. "all box lines", "all accents"). Normally untouched.
%
%   TIER 3: ELEMENTS
%     Every individual visual element has its own color name.
%     By default each inherits from its Tier 2 role.
%     Override any single line to break one element out of its group.
%
% Templates use ONLY Tier 3 names. No template ever references
% a palette color or a role color directly.
%
% ============================================================================

% ----------------------------------------------------------------------------
% TIER 1: PALETTE
% ----------------------------------------------------------------------------
% Uncomment ONE theme block. Comment all others. Recompile.
%
%   pal-bg ........... Page background
%   pal-text ......... Body text, default foreground
%   pal-structure .... Light structural lines (single-line box edges)
%   pal-depth ........ Heavy structural lines (double-line box edges)
%   pal-accent ....... Bullets, branches, markers, active highlights
%   pal-dot .......... Background dot texture fill
%   pal-heading ...... Section title text in box borders
%   pal-subtle ....... De-emphasized text (dates, locations, secondary)
% ----------------------------------------------------------------------------

% --- THEME: Warm Orange (active) -------------------------------------------
% \definecolor{pal-bg}{HTML}{FEFDFA}
% \definecolor{pal-text}{HTML}{333333}
% \definecolor{pal-structure}{HTML}{B8A090}
% \definecolor{pal-depth}{HTML}{9C877A}
% \definecolor{pal-accent}{HTML}{CC5500}
% \definecolor{pal-dot}{HTML}{DDD8D0}
% \definecolor{pal-heading}{HTML}{333333}
% \definecolor{pal-subtle}{HTML}{7A7A7A}

% --- THEME: Cool Blue -------------------------------------------------------
\definecolor{pal-bg}{HTML}{F8FAFC}
\definecolor{pal-text}{HTML}{2D3748}
\definecolor{pal-structure}{HTML}{94A3B8}
\definecolor{pal-depth}{HTML}{64748B}
\definecolor{pal-accent}{HTML}{2563EB}
\definecolor{pal-dot}{HTML}{CBD5E1}
\definecolor{pal-heading}{HTML}{1E293B}
\definecolor{pal-subtle}{HTML}{64748B}

% --- THEME: Monochrome ------------------------------------------------------
% \definecolor{pal-bg}{HTML}{FAFAFA}
% \definecolor{pal-text}{HTML}{262626}
% \definecolor{pal-structure}{HTML}{A3A3A3}
% \definecolor{pal-depth}{HTML}{737373}
% \definecolor{pal-accent}{HTML}{525252}
% \definecolor{pal-dot}{HTML}{D4D4D4}
% \definecolor{pal-heading}{HTML}{262626}
% \definecolor{pal-subtle}{HTML}{737373}

% --- THEME: Forest Green ----------------------------------------------------
% \definecolor{pal-bg}{HTML}{FAFDF7}
% \definecolor{pal-text}{HTML}{2D3A2D}
% \definecolor{pal-structure}{HTML}{A3B89A}
% \definecolor{pal-depth}{HTML}{6B7F63}
% \definecolor{pal-accent}{HTML}{2D6A2D}
% \definecolor{pal-dot}{HTML}{D4DDD0}
% \definecolor{pal-heading}{HTML}{2D3A2D}
% \definecolor{pal-subtle}{HTML}{5A6B5A}

% ----------------------------------------------------------------------------
% TIER 2: ROLES
% ----------------------------------------------------------------------------
% Maps palette → broad functional groups. Override a line here to
% redirect an entire category. For example, to make all box lines
% use the accent color: \colorlet{role-box-light}{pal-accent}
% ----------------------------------------------------------------------------

\colorlet{role-bg}{pal-bg}
\colorlet{role-text}{pal-text}
\colorlet{role-box-light}{pal-structure}
\colorlet{role-box-dark}{pal-depth}
\colorlet{role-accent}{pal-accent}
\colorlet{role-dot}{pal-dot}
\colorlet{role-heading}{pal-heading}
\colorlet{role-subtle}{pal-subtle}

% ----------------------------------------------------------------------------
% TIER 3: ELEMENTS
% ----------------------------------------------------------------------------
% Every visual element in the CV has its own color name.
% Templates reference ONLY these names.
%
% Grouped by region. To override one element, change its \colorlet
% to point at any palette, role, or custom color.
% ----------------------------------------------------------------------------

% --- Page -------------------------------------------------------------------
\colorlet{page-bg}{role-bg}                  % page background
\colorlet{content-bg}{role-bg}               % colorbox masking dots behind text

% --- Body text --------------------------------------------------------------
\colorlet{text-body}{role-text}              % default body text
\colorlet{text-heading}{role-heading}        % section titles in box borders
\colorlet{text-subtle}{role-subtle}          % dates, locations, secondary info
\colorlet{text-bold}{role-text}              % bold text in content

% --- Header (full-width name/title/contact bar) ----------------------------
\colorlet{header-single}{role-box-light}     % ┌ ─ │  (top, left edges)
\colorlet{header-double}{role-box-dark}      % ╖ ║ ╘ ═ ╝  (right, bottom edges)
\colorlet{header-dot}{role-dot}              % · fill inside header
\colorlet{header-name}{role-text}            % name text
\colorlet{header-title}{role-text}           % title text
\colorlet{header-contact}{role-text}         % email, phone, linkedin, location
\colorlet{header-cursor}{role-accent}        % ▌ cursor block after name
\colorlet{header-prompt}{role-accent}        % >_ before title

% --- Left column boxes (Summary, Experience, etc.) -------------------------
\colorlet{left-single}{role-box-light}       % ┌ ─ │  (top, left edges)
\colorlet{left-double}{role-box-dark}        % ╖ ║ ╘ ═ ╝  (right, bottom edges)
\colorlet{left-dot}{role-dot}                % · fill inside box
\colorlet{left-title}{role-heading}          % section title in top border

% --- Right column boxes (Skills, Education, References) --------------------
\colorlet{right-single}{role-box-light}      % ┌ ─ │  (top, left edges)
\colorlet{right-double}{role-box-dark}       % ╖ ║ ╘ ═ ╝  (right, bottom edges)
\colorlet{right-dot}{role-dot}               % · fill inside box
\colorlet{right-title}{role-heading}         % section title in top border

% --- Full-width boxes (span entire grid width) ----------------------------
\colorlet{full-single}{role-box-light}       % ┌ ─ │  (top, left edges)
\colorlet{full-double}{role-box-dark}        % ╖ ║ ╘ ═ ╝  (right, bottom edges)
\colorlet{full-dot}{role-dot}                % · fill inside box
\colorlet{full-title}{role-heading}          % section title in top border

% --- Accent group (bullets, branches, highlights) --------------------------
%     All the same color by default. Override individually to split.
\colorlet{tree-branch}{role-accent}          % ├ │  connectors in treelist
\colorlet{tree-last}{role-accent}            % └  final item connector
\colorlet{skill-bullet}{role-accent}         % ▸  in skill lists
\colorlet{progress-fill}{role-accent}        % ███  filled bar
\colorlet{progress-empty}{role-dot}          % ░░░  empty bar
\colorlet{progress-bracket}{role-text}       % [ ]  bar frame
\colorlet{job-marker}{role-accent}           % ▪ ⬥  in job headers
\colorlet{timeline-dot}{role-accent}         % ·  timeline spine marker
\colorlet{timeline-line}{role-text}          % │  timeline continuation

% --- Structural details inside content -------------------------------------
\colorlet{thin-rule}{role-dot}               % dotted separator line
\colorlet{clean-bullet}{role-dot}            % · in reference/education lists

% ----------------------------------------------------------------------------
% DRAWING SHORTCUTS — used by templates for box-drawing characters
% ----------------------------------------------------------------------------
% Per-region shortcuts keep template code readable while referencing
% the correct Tier 3 element color. Templates never use raw color names.
%
% Naming: H=Header, L=Left, R=Right, F=Full; SL=single-line,
%         DL=double-line, DOT=dot-fill, C=cursor/accent
% ----------------------------------------------------------------------------

% --- Header drawing ---------------------------------------------------------
\newcommand{\HSL}{\color{header-single}}   % ┌ ─ │  (top, left edges)
\newcommand{\HDL}{\color{header-double}}   % ╖ ║ ╘ ═ ╝  (right, bottom edges)
\newcommand{\HDOT}{\color{header-dot}}     % · fill inside header
\newcommand{\HC}{\color{header-cursor}}    % ▌ cursor block, >_ prompt

% --- Left box drawing -------------------------------------------------------
\newcommand{\LSL}{\color{left-single}}     % ┌ ─ │
\newcommand{\LDL}{\color{left-double}}     % ╖ ║ ╘ ═ ╝
\newcommand{\LDOT}{\color{left-dot}}       % · fill

% --- Right box drawing ------------------------------------------------------
\newcommand{\RSL}{\color{right-single}}    % ┌ ─ │
\newcommand{\RDL}{\color{right-double}}    % ╖ ║ ╘ ═ ╝
\newcommand{\RDOT}{\color{right-dot}}      % · fill

% --- Full-width box drawing -------------------------------------------------
\newcommand{\FSL}{\color{full-single}}     % ┌ ─ │
\newcommand{\FDL}{\color{full-double}}     % ╖ ║ ╘ ═ ╝
\newcommand{\FDOT}{\color{full-dot}}       % · fill

% --- Apply page defaults ----------------------------------------------------
\color{text-body}
\pagecolor{page-bg}


% ============================================================================
% §5  SPACING
% ============================================================================
% Every vertical and horizontal gap in the CV as a named value.
% All expressed in grid units (\TPVertModule or \TPHorizModule)
% or em units relative to the current font.
%
% To tighten or loosen the entire CV, adjust these values.
% Templates and components use ONLY these names.
% ============================================================================

% --- Content line height ----------------------------------------------------
%     Body text baselineskip as a multiple of one grid row.
%     1.0 = text sits exactly on grid lines (tight).
%     1.2 = 20% extra leading (current default, readable).
\newcommand{\ContentLeading}{1.2}

% --- Vertical gaps (in grid rows) -------------------------------------------
%     Used via \vspace{\GapFoo\TPVertModule} in components.
%     All values are multiples of one grid row for grid compliance.
\newcommand{\GapAfterSubHead}{0.1}          % below SubHead before content
\newcommand{\GapBeforeSubHead}{0.5}         % above SubHead (between sections)
\newcommand{\GapBeforeDesc}{0.1}            % above italic description block
\newcommand{\GapAfterDesc}{0.5}             % below italic description block
\newcommand{\GapHeaderToContent}{1}          % rows between header bottom and first box
\newcommand{\GapBoxToBox}{1}                % between stacked boxes (LeftBoxGap)
\newcommand{\GapJobToJob}{1}                % between job entries (full row)
\newcommand{\GapSkillCat}{0.5}             % between skill categories
\newcommand{\GapTimelineItem}{0.5}          % between timeline entries

% --- Tree list spacing (grid row multipliers) -------------------------------
%     Used as: \vspace{\TreeTopSkip\TPVertModule} or topsep=\TreeTopSkip\TPVertModule
%     Expressed as grid-row fractions, same pattern as \GapFoo constants above.
\newcommand{\TreeTopSkip}{0.25}            % space above first tree item
\newcommand{\TreeBotSkip}{0.1}             % space below last tree item
\newcommand{\TreeItemSep}{0.1}             % between tree items in explist

% --- Content bleed/nudge (REMOVED) -----------------------------------------
%     No longer needed. The background overlay now covers all dots
%     inside the border deterministically (innerW × bodyRows).
%     Content is positioned via box padding (§6) only.


% ============================================================================
% §6  BOX GEOMETRY
% ============================================================================
% Widths, padding, and column positions. All in grid units.
%
% MASTER PARAMETERS (change these to reshape the layout):
%   \HeaderHeight ......... Rows the header occupies
%   \HeaderContactSep ..... Dot-space gaps between contact items in header
%   \LeftBoxWidth ......... Width of the left column in grid cols
%   \ColumnGap ............ Gap between left and right columns in grid cols
%   \Left/Right/FullBoxPad* Padding inside boxes (cols or rows)
%
% DERIVED VALUES (do not edit — computed from master params):
%   \HeaderWidth .......... = \GridCols (full page width)
%   \ContentStartY ........ = \HeaderHeight + 1 (row where boxes begin)
%   \RightBoxWidth ........ = \GridCols - \LeftBoxWidth - \ColumnGap
%   \RightColX ............ = \LeftBoxWidth + \ColumnGap
%   \FullBoxWidth ......... = \GridCols (full page width)
%   \*ContentWidth ........ = box - padding (no scaling)
% ============================================================================

% --- Header (master) -------------------------------------------------------
\newcommand{\HeaderHeight}{6}               % rows the header occupies
\newcommand{\HeaderContactSep}{3}           % dot-spaces between contact items

% --- Column layout (master) ------------------------------------------------
\newcommand{\LeftBoxWidth}{63}              % left column width in grid cols
\newcommand{\ColumnGap}{1.7}                  % gap between columns in grid cols

% --- Left column padding (master) ------------------------------------------
\newcommand{\LeftBoxPadLeft}{3}             % left padding (cols)
\newcommand{\LeftBoxPadRight}{3}            % right padding (cols)
\newcommand{\LeftBoxPadTop}{1.2}              % top padding (rows inside box)
\newcommand{\LeftBoxPadBot}{1}              % bottom padding (rows inside box)

% --- Right column padding (master) -----------------------------------------
\newcommand{\RightBoxPadLeft}{3}            % left padding (cols)
\newcommand{\RightBoxPadRight}{3}           % right padding (cols)
\newcommand{\RightBoxPadTop}{1.2}             % top padding (rows inside box)
\newcommand{\RightBoxPadBot}{1}             % bottom padding (rows inside box)

% --- Full-width box padding (master) ---------------------------------------
\newcommand{\FullBoxPadLeft}{3}             % left padding (cols)
\newcommand{\FullBoxPadRight}{3}            % right padding (cols)
\newcommand{\FullBoxPadTop}{1.2}              % top padding (rows inside box)
\newcommand{\FullBoxPadBot}{1}              % bottom padding (rows inside box)

% --- Derived layout values (do not edit) ------------------------------------
%     Invariant: LeftBoxWidth + ColumnGap + RightBoxWidth = GridCols
%     This guarantees the left box touches the left margin and the
%     right box touches the right margin, with the gap between them.
\newcommand{\HeaderWidth}{\GridCols}                                        % margin to margin
\newcommand{\ContentStartY}{\fpeval{\HeaderHeight + \GapHeaderToContent}}    % first row below header
\newcommand{\RightBoxWidth}{\fpeval{\GridCols - \LeftBoxWidth - \ColumnGap}}% right column width
\newcommand{\RightColX}{\fpeval{\LeftBoxWidth + \ColumnGap}}               % right column X origin

% --- Derived: full-width box ------------------------------------------------
\newcommand{\FullBoxWidth}{\GridCols}         % spans the entire grid
\pgfmathsetmacro{\tmpFullBoxWidth}{\GridCols} % numeric copy for pgfmath below

% --- Derived content widths (do not edit) -----------------------------------
%     How many grid columns the content minipage spans (box minus padding).
\pgfmathsetmacro{\LeftContentWidth}{(\LeftBoxWidth - \LeftBoxPadLeft - \LeftBoxPadRight) * \ContentWidthScale}
\pgfmathsetmacro{\RightContentWidth}{(\RightBoxWidth - \RightBoxPadLeft - \RightBoxPadRight) * \ContentWidthScale}
\pgfmathsetmacro{\FullContentWidth}{(\tmpFullBoxWidth - \FullBoxPadLeft - \FullBoxPadRight) * \ContentWidthScale}


% ============================================================================
% §7  CONTENT FORMATTING
% ============================================================================
% Lists, tree branches, timelines, helper commands.
% All colors and spacing reference §4 and §5 names.
% ============================================================================

% ----------------------------------------------------------------------------
% §7.1  BASE UTILITIES
% ----------------------------------------------------------------------------

% \Repeat{N}{char} — repeat a character N times
\newcount\repeatcount
\newcommand{\Repeat}[2]{%
    \repeatcount=#1\relax
    \loop\ifnum\repeatcount>0
        #2\advance\repeatcount by -1
    \repeat
}

% \GridVSpace{N} — insert N grid rows of vertical space
\newcommand{\GridVSpace}[1]{%
    \vspace{\fpeval{#1 * \TPVertModule}pt}%
}

% \CountTextCols{text} — count characters (for monospace centering math)
\newcommand{\CountTextCols}[1]{%
    \StrLen{#1}[\measuredColsInt]%
}

% ----------------------------------------------------------------------------
% §7.2  TYPOGRAPHY PRESETS
% ----------------------------------------------------------------------------
% These commands set up the full typographic environment for content
% inside boxes. They enforce grid-aligned baselines so text doesn't
% drift relative to the dot texture.
%
% Usage in templates:
%   \LeftTypography   — inside left box content areas
%   \RightTypography  — inside right box content areas
%   \CurrentTypography — alias set by each box template before rendering,
%                        so treelist/timeline/skilllist use the correct
%                        font size regardless of which column they're in.
% ----------------------------------------------------------------------------

% \CurrentTypography defaults to Left; each box template overrides it.
\newcommand{\CurrentTypography}{\LeftTypography}

\newcommand{\LeftTypography}{%
    \aile\ContentSizeLeft\raggedright%
    \baselineskip=\ContentLeading\TPVertModule\relax%
    \lineskip=0pt\relax%
    \lineskiplimit=0pt\relax%
    \parskip=0pt\relax%
}

\newcommand{\RightTypography}{%
    \aile\ContentSizeRight\raggedright%
    \baselineskip=\ContentLeading\TPVertModule\relax%
    \lineskip=0pt\relax%
    \lineskiplimit=0pt\relax%
    \parskip=0pt\relax%
}

\newcommand{\FullTypography}{%
    \aile\ContentSizeLeft\raggedright%
    \baselineskip=\ContentLeading\TPVertModule\relax%
    \lineskip=0pt\relax%
    \lineskiplimit=0pt\relax%
    \parskip=0pt\relax%
}

% ----------------------------------------------------------------------------
% §7.3  TREE LIST — Bulleted items with │ continuation lines
% ----------------------------------------------------------------------------
% The signature look: ├ / └ connectors with vertical │ continuation
% for multi-line items. Used in experience sections.
%
% Usage:
%   \begin{treelist}
%     \TreeItem{First point that can wrap to multiple lines}
%     \TreeItem{Second point}
%     \TreeLast{Final point — uses └ instead of ├}
%   \end{treelist}
%
% The connector color comes from tree-branch / tree-last (§4 Tier 3).
% Spacing comes from \TreeTopSkip, \TreeBotSkip (§5).
% ----------------------------------------------------------------------------

\newsavebox{\TreeWidthBox}
\newlength{\TreePrefixW}
\setlength{\TreePrefixW}{2em}
\newlength{\ContentPrefixIndent}

\AtBeginDocument{%
    \sbox{\TreeWidthBox}{\mono├─}%
    \ifdim\wd\TreeWidthBox>1pt
        \global\TreePrefixW=\wd\TreeWidthBox\relax
    \fi
    \sbox{\TreeWidthBox}{\mono┌─}%
    \global\ContentPrefixIndent=\dimexpr\wd\TreeWidthBox - 3\TPHorizModule\relax
}

\newsavebox{\TreeMeasureBox}

\newenvironment{treelist}{%
    \par\vspace{\TreeTopSkip\TPVertModule}%
}{%
    \par\vspace{\TreeBotSkip\TPVertModule}%
}

\newcommand{\TreeItem}[1]{%
    \par\noindent\hspace{\ContentPrefixIndent}%
    \sbox{\TreeMeasureBox}{%
        \parbox[t]{\dimexpr\linewidth-\ContentPrefixIndent-\TreePrefixW\relax}{%
            \CurrentTypography #1%
        }%
    }%
    \pgfmathtruncatemacro{\TreeCont}{%
        max(0, ceil((\ht\TreeMeasureBox + \dp\TreeMeasureBox) / (\ContentLeading*\TPVertModule)) - 1)%
    }%
    \hbox to \TreePrefixW{\vtop{%
        \baselineskip=\ContentLeading\TPVertModule\relax%
        \lineskip=0pt\relax%
        \lineskiplimit=0pt\relax%
        \hbox to \TreePrefixW{\CurrentTypography\mono{\color{tree-branch}\vphantom{Xg}├╴ }\hfil}%
        \ifnum\TreeCont>0\relax
            \foreach \n in {1,...,\TreeCont}{%
                \hbox to \TreePrefixW{\mono{\color{tree-branch}│}\hfil}%
            }%
        \fi
    }\hss}%
    \vtop{%
        \hsize=\dimexpr\linewidth-\ContentPrefixIndent-\TreePrefixW\relax
        \CurrentTypography #1\par
    }%
}

\newcommand{\TreeLast}[1]{%
    \par\noindent\hspace{\ContentPrefixIndent}%
    \sbox{\TreeMeasureBox}{%
        \parbox[t]{\dimexpr\linewidth-\ContentPrefixIndent-\TreePrefixW\relax}{%
            \CurrentTypography #1%
        }%
    }%
    \pgfmathtruncatemacro{\TreeCont}{%
        max(0, ceil((\ht\TreeMeasureBox + \dp\TreeMeasureBox) / (\ContentLeading*\TPVertModule)) - 1)%
    }%
    \hbox to \TreePrefixW{\vtop{%
        \baselineskip=\ContentLeading\TPVertModule\relax%
        \lineskip=0pt\relax%
        \lineskiplimit=0pt\relax%
        \hbox to \TreePrefixW{\CurrentTypography\mono{\color{tree-last}\vphantom{Xg}└╴ }\hfil}%
        \ifnum\TreeCont>0\relax
            \foreach \n in {1,...,\TreeCont}{\hbox to \TreePrefixW{\hfil}}%
        \fi
    }\hss}%
    \vtop{%
        \hsize=\dimexpr\linewidth-\ContentPrefixIndent-\TreePrefixW\relax
        \CurrentTypography #1\par
    }%
}

% ----------------------------------------------------------------------------
% §7.4  SKILL LIST — Triangular bullet items for right column
% ----------------------------------------------------------------------------
% Usage:
%   \begin{skilllist}
%     \item PyTorch, HF Transformers
%     \item SapBERT, Lifelines
%   \end{skilllist}
% ----------------------------------------------------------------------------

\newlist{skilllist}{itemize}{1}
\setlist[skilllist]{%
    label={\color{skill-bullet}\mono▸},
    labelwidth=\TreePrefixW,
    labelsep=0pt,
    leftmargin=\dimexpr\ContentPrefixIndent+\TreePrefixW\relax,
    align=left,
    itemsep=0pt,
    parsep=0pt,
    topsep=\TreeTopSkip\TPVertModule
}

% ----------------------------------------------------------------------------
% §7.5  EXP LIST — Tree-branch bullets (legacy, simpler than treelist)
% ----------------------------------------------------------------------------
% For cases where you want ├─ bullets without the multi-line │ logic.
%
% Usage:
%   \begin{explist}
%     \item First point
%     \lastitem Final point  ← uses └─
%   \end{explist}
% ----------------------------------------------------------------------------

\newlist{explist}{itemize}{1}
\setlist[explist]{%
    label={\color{text-body}\mono├─},
    labelwidth=\TreePrefixW,
    labelsep=0pt,
    leftmargin=\dimexpr\ContentPrefixIndent+\TreePrefixW\relax,
    align=left,
    itemsep=\TreeItemSep\TPVertModule,
    parsep=0pt,
    topsep=\TreeTopSkip\TPVertModule,
    after=\vspace{\TreeBotSkip\TPVertModule},
}

\newcommand{\lastitem}{\item[{\color{text-body}\mono└─}]}

% ----------------------------------------------------------------------------
% §7.6  CLEAN LIST — Minimal dot bullets for references/education
% ----------------------------------------------------------------------------

\newlist{cleanlist}{itemize}{1}
\setlist[cleanlist]{%
    label={\color{clean-bullet}\mono·},
    leftmargin=1.2em,
    itemsep=\TreeItemSep\TPVertModule,
    parsep=0pt,
    topsep=\TreeTopSkip\TPVertModule,
}

% ----------------------------------------------------------------------------
% §7.7  TIMELINE — Vertical spine with · markers
% ----------------------------------------------------------------------------
% Used for education history. A dot marks each entry, with │ continuation
% for multi-line content.
%
% Usage:
%   \begin{timeline}
%     \TimelineItem{MSc AI and Digital Health \\ Westminster \\ 2024--2026}
%     \TimelineLast{NQF 6: Product Design \\ Cape Peninsula \\ 2005--2009}
%   \end{timeline}
% ----------------------------------------------------------------------------

\newsavebox{\TLMeasureBox}
\newlength{\TLPrefixW}
\AtBeginDocument{%
    \global\TLPrefixW=\TreePrefixW\relax
}

\newenvironment{timeline}{%
    \par\vspace{\TreeTopSkip\TPVertModule}%
}{%
    \par\vspace{\TreeBotSkip\TPVertModule}%
}

% TimelineItemFirst — first item in a timeline (cancels environment top space)
\newcommand{\TimelineItemFirst}[2]{%
    % #1 = heading text, #2 = body content
    \vspace{-\TreeTopSkip\TPVertModule}%
    \par\noindent\hspace{\ContentPrefixIndent}%
    \sbox{\TLMeasureBox}{%
        \parbox[t]{\dimexpr\linewidth-\ContentPrefixIndent-\TLPrefixW\relax}{%
            \CurrentTypography {\HeadingOutdent\HeadingSize\color{text-bold} #1}\\#2%
        }%
    }%
    \pgfmathtruncatemacro{\TLCont}{%
        max(0, ceil((\ht\TLMeasureBox + \dp\TLMeasureBox) / (\ContentLeading*\TPVertModule)) - 1)%
    }%
    \hbox to \TLPrefixW{\vtop{%
        \baselineskip=\ContentLeading\TPVertModule\relax%
        \lineskip=0pt\relax%
        \lineskiplimit=0pt\relax%
        \hbox to \TLPrefixW{\CurrentTypography\mono{\color{timeline-dot}\vphantom{Xg}·}\hfil}%
        \ifnum\TLCont>0\relax
            \foreach \n in {1,...,\TLCont}{%
                \hbox to \TLPrefixW{\mono{\color{timeline-line}│}\hfil}%
            }%
        \fi
    }\hss}%
    \vtop{%
        \hsize=\dimexpr\linewidth-\ContentPrefixIndent-\TLPrefixW\relax
        \CurrentTypography {\HeadingOutdent\HeadingSize\color{text-bold} #1}\\#2\par
    }%
}

\newcommand{\TimelineItem}[2]{%
    % #1 = heading text, #2 = body content
    \par\noindent\hspace{\ContentPrefixIndent}%
    \sbox{\TLMeasureBox}{%
        \parbox[t]{\dimexpr\linewidth-\ContentPrefixIndent-\TLPrefixW\relax}{%
            \CurrentTypography {\HeadingOutdent\HeadingSize\color{text-bold} #1}\\#2%
        }%
    }%
    \pgfmathtruncatemacro{\TLCont}{%
        max(0, ceil((\ht\TLMeasureBox + \dp\TLMeasureBox) / (\ContentLeading*\TPVertModule)) - 1)%
    }%
    \hbox to \TLPrefixW{\vtop{%
        \baselineskip=\ContentLeading\TPVertModule\relax%
        \lineskip=0pt\relax%
        \lineskiplimit=0pt\relax%
        \hbox to \TLPrefixW{\CurrentTypography\mono{\color{timeline-dot}\vphantom{Xg}·}\hfil}%
        \ifnum\TLCont>0\relax
            \foreach \n in {1,...,\TLCont}{%
                \hbox to \TLPrefixW{\mono{\color{timeline-line}│}\hfil}%
            }%
        \fi
    }\hss}%
    \vtop{%
        \hsize=\dimexpr\linewidth-\ContentPrefixIndent-\TLPrefixW\relax
        \CurrentTypography {\HeadingOutdent\HeadingSize\color{text-bold} #1}\\#2\par
    }%
}

\newcommand{\TimelineLast}[2]{%
    % #1 = heading text, #2 = body content
    \par\noindent\hspace{\ContentPrefixIndent}%
    \sbox{\TLMeasureBox}{%
        \parbox[t]{\dimexpr\linewidth-\ContentPrefixIndent-\TLPrefixW\relax}{%
            \CurrentTypography {\HeadingOutdent\HeadingSize\color{text-bold} #1}\\#2%
        }%
    }%
    \pgfmathtruncatemacro{\TLCont}{%
        max(0, ceil((\ht\TLMeasureBox + \dp\TLMeasureBox) / (\ContentLeading*\TPVertModule)) - 1)%
    }%
    \hbox to \TLPrefixW{\vtop{%
        \baselineskip=\ContentLeading\TPVertModule\relax%
        \lineskip=0pt\relax%
        \lineskiplimit=0pt\relax%
        \hbox to \TLPrefixW{\CurrentTypography\mono{\color{timeline-dot}\vphantom{Xg}·}\hfil}%
        \ifnum\TLCont>0\relax
            \foreach \n in {1,...,\TLCont}{\hbox to \TLPrefixW{\hfil}}%
        \fi
    }\hss}%
    \vtop{%
        \hsize=\dimexpr\linewidth-\ContentPrefixIndent-\TLPrefixW\relax
        \CurrentTypography {\HeadingOutdent\HeadingSize\color{text-bold} #1}\\#2\par
    }%
}

% ----------------------------------------------------------------------------
% §7.8  COMPONENT HELPERS
% ----------------------------------------------------------------------------
% Reusable formatting commands for content inside components.
% All colors and spacing pull from centralised values.
% ----------------------------------------------------------------------------

% \HeadingOutdent — pull subheading-style text half a grid column left
%   Used by \SubHead, \SkillCat, \TimelineItem, \TimelineLast.
\newcommand{\HeadingOutdent}{\hspace{-0.75\TPHorizModule}}

% SubHeadFirst — first subheading in a box (no top gap)
\newcommand{\SubHeadFirst}[1]{%
    {\HeadingOutdent\HeadingSize\color{text-bold} #1}%
    \vspace{\GapAfterSubHead\TPVertModule}%
}

% SubHead — bold subsection title (e.g. "Platform Architecture & Orchestration")
\newcommand{\SubHead}[1]{%
    \vspace{\GapBeforeSubHead\TPVertModule}%
    {\HeadingOutdent\HeadingSize\color{text-bold} #1}%
    \vspace{\GapAfterSubHead\TPVertModule}%
}

% Role — job/project title with right-aligned date
%   \Role{Scientific Literature Processing Platform}{2024--Present}
\newcommand{\Role}[2]{%
    {\bfseries\color{text-bold} #1}\hfill%
    {\SecondarySize\color{text-subtle} #2}%
}

% Org — company/institution with right-aligned location
%   \Org{University of Westminster}{London, UK}
\newcommand{\Org}[2]{%
    {\SecondarySize\color{text-body} #1\hfill\color{text-subtle} #2}%
}

% Desc — italic description paragraph
%   \Desc{Integrated platform for ingesting...}
\newcommand{\Desc}[1]{%
    \vspace{\GapBeforeDesc\TPVertModule}%
    {\SecondarySize\itshape\color{text-body} #1}%
    \vspace{\GapAfterDesc\TPVertModule}%
}

% SkillCatFirst — first category in a box (cancels any top space)
\newcommand{\SkillCatFirst}[1]{%
    \vspace{-\TreeTopSkip\TPVertModule}%
    {\HeadingOutdent\HeadingSize\color{text-bold} #1}%
}

% SkillCat — bold category label in skills section
%   \SkillCat{GPU / Compute}
\newcommand{\SkillCat}[1]{%
    {\HeadingOutdent\HeadingSize\color{text-bold} #1}%
}

% ThinRule — dotted separator
\newcommand{\ThinRule}{%
    \vspace{0.25\TPVertModule}%
    {\color{thin-rule}\mono\Repeat{20}{·}}%
    \vspace{0.25\TPVertModule}%
}

% ProgressBar — terminal-style progress indicator with optional label
%   \ProgressBar{70}{Some label}  →  [███████░░░] Some label
%   \ProgressBar{70}{}            →  [███████░░░]
\newcommand{\ProgressBar}[2]{%
    \pgfmathtruncatemacro{\filled}{floor(#1 / 10)}%
    \pgfmathtruncatemacro{\empty}{10 - \filled}%
    {\mono\SecondarySize%
        {\color{progress-bracket}[}%
        {\color{progress-fill}\Repeat{\filled}{█}}%
        {\color{progress-empty}\Repeat{\empty}{░}}%
        {\color{progress-bracket}]}%
    }%
    \if\relax\detokenize{#2}\relax\else%
        {\SecondarySize\color{text-body}\ #2}%
    \fi%
}

% JobSep — vertical space between job entries
\newcommand{\JobSep}{\vspace{\GapJobToJob\TPVertModule}}

% JobHead — full job header block
%   \JobHead{Role Title}{Company Name}{2020--2023}{London, UK}
\newcommand{\JobHead}[4]{%
    {\HeadingOutdent\HeadingSize\color{text-bold} #1}%
    \hfill{\SecondarySize{\color{job-marker}▪}\,{\color{text-subtle}#4}}\\
    {\HeadingOutdent\SecondarySize\color{text-body} #2%
    \hfill{\color{job-marker}⬥}\,{\color{text-subtle}#3}}%
}


% ============================================================================
% §8  DEBUG
% ============================================================================

\newcommand{\ShowGridInfo}{%
    Grid: \GridCols\ cols × \GridRows\ rows |
    Cell: \CellWidth mm × \CellHeight mm |
    Margin: \MarginH mm (H) × \MarginV mm (V) |
    Font: \GridFontSize pt%
}

\newcommand{\DebugGrid}{%
    \begin{textblock}{\GridCols}(0,\fpeval{\GridRows - 1})%
        \tiny\mono\color{text-subtle} \ShowGridInfo%
    \end{textblock}%
}

% ============================================================================
% §9  CONTACT DATA — shared macros populated by components/contact.tex
% ============================================================================
% Each \Contact* command stores its argument into a macro.
% canvas.tex inputs the contact file, then feeds the macros to \SetCVHeader.
% The ATS generator (scripts/generate_ats.py) parses the same file via regex.
% ============================================================================
\newcommand{\ContactName}[1]{\gdef\StoredContactName{#1}}
\newcommand{\ContactTitle}[1]{\gdef\StoredContactTitle{#1}}
\newcommand{\ContactEmail}[1]{\gdef\StoredContactEmail{#1}}
\newcommand{\ContactPhone}[1]{\gdef\StoredContactPhone{#1}}
\newcommand{\ContactLinkedIn}[1]{\gdef\StoredContactLinkedIn{#1}}
\newcommand{\ContactGitHub}[1]{\gdef\StoredContactGitHub{#1}}
\newcommand{\ContactLocation}[1]{\gdef\StoredContactLocation{#1}}
\newcommand{\ContactFullCV}[1]{\gdef\StoredContactFullCV{#1}}

% Defaults (in case a field is missing from the contact file)
\ContactName{Your Name}
\ContactTitle{Your Title}
\ContactEmail{email@example.com}
\ContactPhone{+00 000 000 0000}
\ContactLinkedIn{linkedin.com/in/you}
\ContactGitHub{github.com/you}
\ContactLocation{City, Country}
\ContactFullCV{https://example.com/cv.pdf}

\endinput