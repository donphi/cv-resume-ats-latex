% ============================================================================
% TEMPLATES/LEFTBOX.TEX — Left Column Auto-Height Boxes
% ============================================================================
%
% ZERO hardcoded values. All parameters come from preamble.tex:
%   §6  \LeftBoxWidth, \LeftBoxPad*, \LeftContentWidth
%   §4  \LSL, \LDL, \LDOT  (left box color shortcuts)
%   §7  \LeftTypography, \Repeat
%   §4  content-bg  (Tier 3 color for content background)
%
% SETUP (in canvas.tex):
%   \LeftBoxInit{0}{\ContentStartY}
%
% USAGE:
%   \LeftBox{SECTION TITLE}{components/your-content.tex}
%   \LeftBoxGap{1}
%   \LeftBox{NEXT TITLE}{components/next-content.tex}
%
% HOW IT WORKS:
%   - Box origin (0,0) is TOP-LEFT corner of the drawable area
%   - X increases rightward, Y increases downward
%   - Each box auto-measures content height
%   - Y position auto-updates after each box
%
% STRUCTURAL CONSTANTS (derived from box-drawing pattern, not params):
%   7 = border/dot/space char count in title row: ┌(1) ─(1) ·(1) space(1) TITLE space(1) ·(1) ╖(1)
%   2 = left + right border columns (│ and ║)
%
% ============================================================================

% ----------------------------------------------------------------------------
% POSITION TRACKING (internal — set via \LeftBoxInit)
% ----------------------------------------------------------------------------
\newcommand{\LeftPosX}{0}               % Current X position
\newcommand{\LeftPosY}{0}               % Current Y position

% Initialize position — CALL THIS FIRST in canvas.tex
\newcommand{\LeftBoxInit}[2]{%
    % #1 = X position (grid columns from left)
    % #2 = Y position (grid rows from top)
    \xdef\LeftPosX{#1}%
    \xdef\LeftPosY{#2}%
}

% ----------------------------------------------------------------------------
% CONTENT MEASUREMENT BOX (internal)
% ----------------------------------------------------------------------------
\newsavebox{\LeftMeasureBox}

% ----------------------------------------------------------------------------
% MAIN BOX COMMAND
% ----------------------------------------------------------------------------
% Usage: \LeftBox{TITLE}{content-file.tex}
%
\newcommand{\LeftBox}[2]{%
    % #1 = TITLE (appears in top border)
    % #2 = content file path (e.g., components/summary.tex)
    %
    % =========================================================================
    % STEP 1: MEASURE CONTENT HEIGHT
    % =========================================================================
    \renewcommand{\CurrentTypography}{\LeftTypography}%
    \savebox{\LeftMeasureBox}{%
        \begin{minipage}{\LeftContentWidth\TPHorizModule}%
            \LeftTypography%
            \input{#2}%
        \end{minipage}%
    }%
    %
    % Calculate rows needed
    \pgfmathsetmacro{\rawHeightPt}{\ht\LeftMeasureBox + \dp\LeftMeasureBox}%
    \pgfmathtruncatemacro{\contentRows}{ceil(\rawHeightPt / \TPVertModule)}%
    %
    % Total box height = top border + padding + content + padding + bottom border
    \pgfmathtruncatemacro{\boxRows}{1 + \LeftBoxPadTop + \contentRows + \LeftBoxPadBot + 1}%
    \pgfmathtruncatemacro{\bodyRows}{\boxRows - 2}%
    \pgfmathtruncatemacro{\innerW}{\LeftBoxWidth - 2}%
    %
    % Debug output (visible in .log file)
    \typeout{LEFTBOX [\detokenize{#1}]: rawPt=\rawHeightPt, contentRows=\contentRows, boxRows=\boxRows}%

    %
    % =========================================================================
    % STEP 2: DRAW BOX FRAME
    % =========================================================================
    \StrLen{#1}[\titleLen]%
    \pgfmathtruncatemacro{\dashCount}{\LeftBoxWidth - 7 - \titleLen}%
    %
    \begin{textblock}{\LeftBoxWidth}(\LeftPosX,\LeftPosY)%
        \mono%
        % Kill all automatic spacing — critical for deterministic heights
        \baselineskip=0pt\relax%
        \lineskip=0pt\relax%
        \lineskiplimit=0pt\relax%
        \parskip=0pt\relax%
        \offinterlineskip%
        %
        % Top border with title
        \vbox to \TPVertModule{\vss\hbox{{\LSL┌─·} \textbf{#1} {\LSL·\Repeat{\dashCount}{─}}{\LDL╖}}\vss}%
        % Body rows (dot-filled)
        \foreach \n in {1,...,\bodyRows}{%
            \vbox to \TPVertModule{\vss\hbox{{\LSL│}{\LDOT\Repeat{\innerW}{·}}{\LDL║}}\vss}%
        }%
        % Bottom border
        \vbox to \TPVertModule{\vss\hbox{{\LDL╘\Repeat{\innerW}{═}╝}}\vss}%
    \end{textblock}%
    %
    % =========================================================================
    % STEP 3: PLACE CONTENT (two layers: background overlay + text)
    % =========================================================================
    %
    % Layer 1: Background rectangle covering dots inside the border,
    %          inset by 1 grid cell on each side so 1 dot column/row stays visible.
    \pgfmathsetmacro{\bgX}{\LeftPosX + 2}%
    \pgfmathsetmacro{\bgY}{\LeftPosY + 2}%
    \pgfmathtruncatemacro{\bgW}{\innerW - 3}%
    \pgfmathtruncatemacro{\bgH}{\bodyRows - 3}%
    \begin{textblock}{\bgW}(\bgX,\bgY)%
        \colorbox{content-bg}{\makebox[\bgW\TPHorizModule][l]{\rule{0pt}{\bgH\TPVertModule}}}%
    \end{textblock}%
    %
    % Layer 2: Content text, positioned with padding offsets on top of the bg.
    \pgfmathsetmacro{\contentX}{\LeftPosX + \LeftBoxPadLeft}%
    \pgfmathsetmacro{\contentY}{\LeftPosY + 1 + \LeftBoxPadTop}%
    \begin{textblock}{\LeftContentWidth}(\contentX,\contentY)%
        \begin{minipage}{\LeftContentWidth\TPHorizModule}%
            \LeftTypography%
            \input{#2}%
        \end{minipage}%
    \end{textblock}%
    %
    % =========================================================================
    % STEP 4: UPDATE Y FOR NEXT BOX
    % =========================================================================
    \pgfmathsetmacro{\newY}{\LeftPosY + \boxRows}%
    \xdef\LeftPosY{\newY}%
}

% ----------------------------------------------------------------------------
% GAP COMMAND
% ----------------------------------------------------------------------------
% Usage: \LeftBoxGap{1}     % Add 1 row gap
%        \LeftBoxGap{0.5}   % Add half row gap
%
\newcommand{\LeftBoxGap}[1]{%
    \pgfmathsetmacro{\newY}{\LeftPosY + #1}%
    \xdef\LeftPosY{\newY}%
}

% ----------------------------------------------------------------------------
% DEBUG: Show current Y position
% ----------------------------------------------------------------------------
% Usage: \LeftBoxDebug      % Shows red "Y=xxx" at current position
%
\newcommand{\LeftBoxDebug}{%
    \begin{textblock}{20}(\LeftPosX,\LeftPosY)%
        \tiny\mono\color{red}Y=\LeftPosY%
    \end{textblock}%
}

\endinput
