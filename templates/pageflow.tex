% ============================================================================
% TEMPLATES/PAGEFLOW.TEX — Multi-Page Pagination Engine
% ============================================================================
%
% ZERO hardcoded values. All parameters come from preamble.tex:
%   §2  \GridRows               (total rows in the drawable area)
%   §6  \ContentStartY          (first row below the header)
%
% PURPOSE:
%   Handles everything needed when the CV spans more than one page:
%   1. Stores header data once, so it can be repeated on every page.
%   2. Provides \CVPageBreak which starts a new page, re-draws the
%      header, and resets the left/right column Y cursors so boxes
%      continue from the correct row.
%   3. Exposes \MaxPageY — the last usable grid row — so future
%      overflow detection can compare column Y positions against it.
%
% LOAD ORDER:
%   This file must be loaded AFTER the template files (header.tex,
%   leftbox.tex, rightbox.tex) because it uses \CVHeader, \LeftPosY,
%   and \RightPosY which those templates define.
%   canvas.tex loads it in the correct order automatically.
%
% USAGE (in canvas.tex):
%   \SetCVHeader{Name}{Title}{email}{phone}{linkedin}{location}
%   ... place boxes on page 1 ...
%   \CVPageBreak
%   ... place boxes on page 2 (header auto-repeats, Y resets) ...
%
% ============================================================================

% --- Overflow boundary (derived) --------------------------------------------
%     Last usable grid row. Compare \LeftPosY or \RightPosY against this
%     to detect when content would overflow the page.
\newcommand{\MaxPageY}{\GridRows}

% --- Store header arguments for repetition ----------------------------------
%     Called once in canvas.tex. Stores all 6 header fields as global macros
%     so \RepeatHeader can re-draw the header on subsequent pages.
\newcommand{\SetCVHeader}[6]{%
    \gdef\StoredName{#1}%
    \gdef\StoredTitle{#2}%
    \gdef\StoredEmail{#3}%
    \gdef\StoredPhone{#4}%
    \gdef\StoredLinkedin{#5}%
    \gdef\StoredLocation{#6}%
}

% --- Replay the stored header -----------------------------------------------
%     Internal helper. Always placed at grid origin (0,0).
\newcommand{\RepeatHeader}{%
    \CVHeader{0}{0}%
        {\StoredName}{\StoredTitle}%
        {\StoredEmail}{\StoredPhone}%
        {\StoredLinkedin}{\StoredLocation}%
}

% --- Page break with header repetition + Y reset ----------------------------
%     Inserts a page break, re-draws the header at the top of the new page,
%     and resets ALL column Y cursors to \ContentStartY so boxes placed
%     after this command start in the correct row.
\newcommand{\CVPageBreak}{%
    \null\newpage\null%
    \RepeatHeader
    \xdef\LeftPosY{\ContentStartY}%
    \xdef\RightPosY{\ContentStartY}%
    \xdef\FullPosY{\ContentStartY}%
}

\endinput
