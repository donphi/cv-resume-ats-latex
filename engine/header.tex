% ============================================================================
% TEMPLATES/HEADER.TEX — Full-Width CV Header
% ============================================================================
%
% ZERO hardcoded values. All parameters come from preamble.tex:
%   §6  \HeaderWidth, \HeaderHeight, \HeaderContactSep
%   §4  \HSL, \HDL, \HDOT, \HC  (header color shortcuts)
%   §7  \CountTextCols, \Repeat
%
% USAGE (in generated/canvas.tex):
%   \CVHeader{X}{Y}
%       {\StoredName}{\StoredTitle}
%       {\StoredEmail}{\StoredPhone}
%       {\StoredLinkedin}{\StoredLocation}
%
% STRUCTURAL CONSTANTS (derived from box-drawing pattern, not params):
%   6 = border/dot/space char count in name row:  ┌(1) ·(1) space(1) NAME ▌(1) ·(1) ╖(1) = 6 fixed
%   6 = border/dot/space char count in title row: │(1) ·(1) space(1) >_(2) TITLE space(1) ·(1) ║(1) — but see titlePadLeft/Right
%   2 = left + right border columns (│ and ║)
%   4 = per-contact-item fixed chars: space(1) ITEM space(1) + dot-space overlap
%
% ============================================================================

% ----------------------------------------------------------------------------
% MAIN HEADER COMMAND
% ----------------------------------------------------------------------------
\newcommand{\CVHeader}[8]{%
    % #1 = X, #2 = Y
    % #3 = NAME, #4 = TITLE
    % #5 = email, #6 = phone, #7 = linkedin, #8 = location
    %
    % =========================================================================
    % STEP 1: COUNT ALL TEXT LENGTHS (character count for monospace)
    % =========================================================================
    %
    \CountTextCols{#3}%
    \pgfmathtruncatemacro{\nameLen}{\measuredColsInt}%
    %
    \CountTextCols{#4}%
    \pgfmathtruncatemacro{\titleLen}{\measuredColsInt}%
    %
    \CountTextCols{#5}%
    \pgfmathtruncatemacro{\emailLen}{\measuredColsInt}%
    %
    \CountTextCols{#6}%
    \pgfmathtruncatemacro{\phoneLen}{\measuredColsInt}%
    %
    \CountTextCols{#7}%
    \pgfmathtruncatemacro{\linkedinLen}{\measuredColsInt}%
    %
    \CountTextCols{#8}%
    \pgfmathtruncatemacro{\locationLen}{\measuredColsInt}%
    %
    % =========================================================================
    % STEP 2: DEBUG OUTPUT (visible in .log file)
    % =========================================================================
    \typeout{==================================================}%
    \typeout{HEADER DEBUG: HeaderWidth = \HeaderWidth}%
    \typeout{HEADER DEBUG: NAME = #3}%
    \typeout{HEADER DEBUG: NAME char count = \nameLen}%
    \typeout{HEADER DEBUG: TITLE = #4}%
    \typeout{HEADER DEBUG: TITLE char count = \titleLen}%
    \typeout{HEADER DEBUG: email=\emailLen, phone=\phoneLen, linkedin=\linkedinLen, location=\locationLen}%
    %
    % =========================================================================
    % STEP 3: CALCULATE DASH COUNTS
    % =========================================================================
    %
    % NAME LINE: ┌(1) + dashes + ·(1) + space(1) + NAME + ▌(1) + ·(1) + dashes + ╖(1)
    % Fixed = 6 chars, so dashes = HeaderWidth - 6 - nameLen
    %
    \pgfmathtruncatemacro{\nameDashTotal}{\HeaderWidth - 6 - \nameLen}%
    \pgfmathtruncatemacro{\nameDashLeft}{\nameDashTotal / 2}%
    \pgfmathtruncatemacro{\nameDashRight}{\nameDashTotal - \nameDashLeft}%
    %
    \typeout{HEADER DEBUG: NAME dashes total=\nameDashTotal, left=\nameDashLeft, right=\nameDashRight}%
    \typeout{HEADER DEBUG: NAME check: 1+\nameDashLeft+1+1+\nameLen+1+1+\nameDashRight+1 = \the\numexpr1+\nameDashLeft+1+1+\nameLen+1+1+\nameDashRight+1\relax}%
    %
    % TITLE LINE: ├(1) + dashes + ·(1) + space(1) + TITLE + space(1) + ·(1) + dashes + ┤(1)
    % Fixed = 6 chars
    %
    \pgfmathtruncatemacro{\titleDashTotal}{\HeaderWidth - 6 - \titleLen}%
    \pgfmathtruncatemacro{\titleDashLeft}{\titleDashTotal / 2}%
    \pgfmathtruncatemacro{\titleDashRight}{\titleDashTotal - \titleDashLeft}%
    %
    \typeout{HEADER DEBUG: TITLE dashes total=\titleDashTotal, left=\titleDashLeft, right=\titleDashRight}%
    \typeout{HEADER DEBUG: TITLE check: 1+\titleDashLeft+1+1+\titleLen+1+1+\titleDashRight+1 = \the\numexpr1+\titleDashLeft+1+1+\titleLen+1+1+\titleDashRight+1\relax}%
    %
    % CONTACT LINE: │(1) + outerL + · email · + sep + · phone · + sep + · linkedin · + sep + · location · + outerR + ║(1)
    % Each item block: space(1) + ITEM + space(1) = 2 + itemLen
    %
    % The separator between items and the outer dot-padding are AUTO-COMPUTED
    % so the line never overflows. \HeaderContactSep (preamble §6) is the
    % preferred maximum — the actual value is clamped to what fits.
    %
    % 1. Text-only width (items + surrounding spaces, no separators)
    \pgfmathtruncatemacro{\contactTextOnly}{%
        (2 + \emailLen) + (2 + \phoneLen) + (2 + \linkedinLen) + (2 + \locationLen)%
    }%
    % 2. Available columns for dot zones (3 inner seps + 2 outer pads)
    \pgfmathtruncatemacro{\contactAvail}{\HeaderWidth - 2 - \contactTextOnly}%
    % 3. Inner sep = min(preferred, floor(available / 5))  — never negative
    \pgfmathtruncatemacro{\contactSep}{max(0, min(\HeaderContactSep, \contactAvail / 5))}%
    % 4. Outer dots = whatever remains after the 3 inner seps
    \pgfmathtruncatemacro{\contactOuter}{max(0, \contactAvail - 3 * \contactSep)}%
    \pgfmathtruncatemacro{\contactDashLeft}{\contactOuter / 2}%
    \pgfmathtruncatemacro{\contactDashRight}{\contactOuter - \contactDashLeft}%
    %
    \typeout{HEADER DEBUG: CONTACT textOnly=\contactTextOnly, avail=\contactAvail, sep=\contactSep\space(preferred \HeaderContactSep), outerL=\contactDashLeft, outerR=\contactDashRight}%
    %
    % 5. Warn if contact text cannot fit even with zero separators
    \ifnum\contactAvail<0\relax
        \PackageWarning{header}{Contact line overflows by \the\numexpr-\contactAvail\relax\space chars!^^J%
            Shorten contact details or use a wider margin (lower margin value in contact.yaml).}%
    \fi
    %
    \pgfmathtruncatemacro{\innerW}{\HeaderWidth - 2}%
    \typeout{HEADER DEBUG: innerW = \innerW}%
    \typeout{==================================================}%
    %
    % =========================================================================
    % STEP 4: DRAW HEADER
    % =========================================================================
    \begin{textblock}{\HeaderWidth}(#1,#2)%
        \mono%
        \baselineskip=0pt\relax%
        \lineskip=0pt\relax%
        \lineskiplimit=0pt\relax%
        \parskip=0pt\relax%
        \offinterlineskip%
        %
        % ROW 0: Name
        \vbox to \TPVertModule{\vss\hbox{%
            {\HSL┌\Repeat{\nameDashLeft}{─}}· \textbf{\MakeUppercase{#3}}{\HC▌}·{\HSL\Repeat{\nameDashRight}{─}}{\HDL╖}%
        }\vss}%
        %
        % ROW 1: Dot-filled empty
        \vbox to \TPVertModule{\vss\hbox{{\HSL│}{\HDOT\Repeat{\innerW}{·}}{\HDL║}}\vss}%
        %
        % ROW 2: Title (centered, dot-filled)
        \pgfmathtruncatemacro{\titlePadLeft}{(\innerW - \titleLen - 4) / 2}%
        \pgfmathtruncatemacro{\titlePadRight}{\innerW - \titleLen - 4 - \titlePadLeft}%
        \vbox to \TPVertModule{\vss\hbox{%
            {\HSL│}{\HDOT\Repeat{\titlePadLeft}{·}}{ }{\HC>\_}\textbf{\MakeUppercase{#4}}{ }{\HDOT\Repeat{\titlePadRight}{·}}{\HDL║}%
        }\vss}%
        %
        % ROW 3: Dot-filled empty
        \vbox to \TPVertModule{\vss\hbox{{\HSL│}{\HDOT\Repeat{\innerW}{·}}{\HDL║}}\vss}%
        %
        % ROW 4: Contact info (dot-filled, auto-spaced)
        \vbox to \TPVertModule{\vss\hbox{%
            {\HSL│}{\HDOT\Repeat{\contactDashLeft}{·}}%
            { }#5{ }{\HDOT\Repeat{\contactSep}{·}}%
            { }#6{ }{\HDOT\Repeat{\contactSep}{·}}%
            { }#7{ }{\HDOT\Repeat{\contactSep}{·}}%
            { }#8{ }%
            {\HDOT\Repeat{\contactDashRight}{·}}{\HDL║}%
        }\vss}%
        %
        % ROW 5: Bottom border
        \vbox to \TPVertModule{\vss\hbox{{\HDL╘\Repeat{\innerW}{═}╝}}\vss}%
    \end{textblock}%
}

\endinput
