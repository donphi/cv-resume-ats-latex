% ============================================================================
% TEMPLATES/HEADER_CRT.TEX — CRT Block-Font Name Header
% ============================================================================
%
% ZERO hardcoded values. All parameters come from preamble.tex and
% generated/header_name.tex (pre-computed by scripts/header.py).
%
% From preamble.tex:
%   §6  \HeaderWidth, \HeaderHeight, \HeaderContactSep
%   §4  \HSL, \HDL, \HDOT, \HC, \HSHADE  (header color shortcuts)
%   §7  \CountTextCols, \Repeat
%
% From generated/header_name.tex (pre-computed — NO TeX-side name math):
%   \HeaderNameRowA .. \HeaderNameRowD  (4 rows of ASCII art)
%   \HeaderNameWidth                    (display width in columns)
%   \NameDashLeft, \NameDashRight       (dash counts for row 0)
%   \NameDotLeft, \NameDotRight         (dot counts for rows 1-3)
%
% STRUCTURE (6 rows):
%   Row 0: ┌────· NAME_ROW_A ·────╖       (top border + first name row)
%   Row 1: │····· NAME_ROW_B ·····║       (dot fill + name row 2)
%   Row 2: │····· NAME_ROW_C ·····║       (dot fill + name row 3)
%   Row 3: │····· NAME_ROW_D ·····║       (dot fill + name row 4 / shadow)
%   Row 4: │···· contact info ····║       (contact line, same as classic)
%   Row 5: ╘══════════════════════╝      (bottom border)
%
% ============================================================================

% --- Load pre-rendered name + pre-computed centering values ---
% header_name.tex — not used for classic theme


% ----------------------------------------------------------------------------
% MAIN HEADER COMMAND
% ----------------------------------------------------------------------------
\newcommand{\CVHeader}[8]{%
    % #1 = X, #2 = Y
    % #3 = NAME (unused — pre-rendered), #4 = TITLE (unused for crt)
    % #5 = email, #6 = phone, #7 = linkedin, #8 = location
    %
    % =========================================================================
    % STEP 1: CONTACT LINE MATH (identical to classic header.tex)
    % =========================================================================
    %
    \CountTextCols{#5}%
    \pgfmathtruncatemacro{\emailLen}{\measuredColsInt}%
    %
    \CountTextCols{#6}%
    \pgfmathtruncatemacro{\phoneLen}{\measuredColsInt}%
    %
    \CountTextCols{#7}%
    \pgfmathtruncatemacro{\linkedinLen}{\measuredColsInt}%
    %
    \CountTextCols{#8}%
    \pgfmathtruncatemacro{\locationLen}{\measuredColsInt}%
    %
    % Contact row: │(1) + outerL + items + outerR + ║(1)
    % Each item: space(1) + ITEM + space(1) = 2 + itemLen
    %
    \pgfmathtruncatemacro{\contactTextOnly}{%
        (2 + \emailLen) + (2 + \phoneLen) + (2 + \linkedinLen) + (2 + \locationLen)%
    }%
    \pgfmathtruncatemacro{\contactAvail}{\HeaderWidth - 2 - \contactTextOnly}%
    \pgfmathtruncatemacro{\contactSep}{max(0, min(\HeaderContactSep, \contactAvail / 5))}%
    \pgfmathtruncatemacro{\contactOuter}{max(0, \contactAvail - 3 * \contactSep)}%
    \pgfmathtruncatemacro{\contactDashLeft}{\contactOuter / 2}%
    \pgfmathtruncatemacro{\contactDashRight}{\contactOuter - \contactDashLeft}%
    %
    \pgfmathtruncatemacro{\innerW}{\HeaderWidth - 2}%
    %
    % =========================================================================
    % STEP 2: DEBUG
    % =========================================================================
    \typeout{==================================================}%
    \typeout{HEADER CRT DEBUG: HeaderWidth = \HeaderWidth}%
    \typeout{HEADER CRT DEBUG: NameDashLeft = \NameDashLeft, NameDashRight = \NameDashRight}%
    \typeout{HEADER CRT DEBUG: NameDotLeft = \NameDotLeft, NameDotRight = \NameDotRight}%
    \typeout{HEADER CRT DEBUG: contactSep = \contactSep, contactDashLeft = \contactDashLeft, contactDashRight = \contactDashRight}%
    \typeout{==================================================}%
    %
    % =========================================================================
    % STEP 3: DRAW HEADER
    % =========================================================================
    \begin{textblock}{\HeaderWidth}(#1,#2)%
        \mono%
        \baselineskip=0pt\relax%
        \lineskip=0pt\relax%
        \lineskiplimit=0pt\relax%
        \parskip=0pt\relax%
        \offinterlineskip%
        %
        % ROW 0: Top border + Name row A
        % Block/shade chars (█▀▄░) are 11.25pt tall (1.25× grid cell).
        % \FullHRow lets each row use its natural height — no clipping.
        \FullHRow{%
            {\HSL┌\Repeat{\NameDashLeft}{─}}·%
            ~{\HSHADE\HeaderNameRowA}~%
            ·{\HSL\Repeat{\NameDashRight}{─}}{\HDL╖}%
        }%
        %
        % ROW 1: Name row B (dot-filled)
        \FullHRow{%
            {\HSL│}{\HDOT\Repeat{\NameDotLeft}{·}}%
            ~{\HSHADE\HeaderNameRowB}~%
            {\HDOT\Repeat{\NameDotRight}{·}}{\HDL║}%
        }%
        %
        % ROW 2: Name row C (dot-filled)
        \FullHRow{%
            {\HSL│}{\HDOT\Repeat{\NameDotLeft}{·}}%
            ~{\HSHADE\HeaderNameRowC}~%
            {\HDOT\Repeat{\NameDotRight}{·}}{\HDL║}%
        }%
        %
        % ROW 3: Name row D / shadow (dot-filled)
        \FullHRow{%
            {\HSL│}{\HDOT\Repeat{\NameDotLeft}{·}}%
            ~{\HSHADE\HeaderNameRowD}~%
            {\HDOT\Repeat{\NameDotRight}{·}}{\HDL║}%
        }%
        %
        % ROW 4: Contact info (dot-filled, auto-spaced — identical to classic)
        \vbox to \TPVertModule{\vss\hbox{%
            {\HSL│}{\HDOT\Repeat{\contactDashLeft}{·}}%
            { }#5{ }{\HDOT\Repeat{\contactSep}{·}}%
            { }#6{ }{\HDOT\Repeat{\contactSep}{·}}%
            { }#7{ }{\HDOT\Repeat{\contactSep}{·}}%
            { }#8{ }%
            {\HDOT\Repeat{\contactDashRight}{·}}{\HDL║}%
        }\vss}%
        %
        % ROW 5: Bottom border
        \vbox to \TPVertModule{\vss\hbox{{\HDL╘\Repeat{\innerW}{═}╝}}\vss}%
    \end{textblock}%
}

\endinput
