% ============================================================================
% TEMPLATES/FULLBOX.TEX — Full-Width Auto-Height Boxes
% ============================================================================
%
% ZERO hardcoded values. All parameters come from preamble.tex:
%   §6  \FullBoxWidth (= \GridCols), \FullBoxPad*, \FullContentWidth
%   §4  \FSL, \FDL, \FDOT  (full-width box color shortcuts)
%   §7  \FullTypography, \Repeat
%   §4  content-bg  (Tier 3 color for content background)
%
% PURPOSE:
%   A single box that spans the entire grid width (\GridCols columns).
%   Use this when a section should not be split into left/right columns —
%   for example, a standalone skills table, a full-width summary, or any
%   content that benefits from the full page width.
%
% SETUP (in canvas.tex):
%   \FullBoxInit{0}{\ContentStartY}
%
% USAGE:
%   \FullBox{SECTION TITLE}{generated/your-content.tex}
%   \FullBoxGap{\GapBoxToBox}
%   \FullBox{NEXT TITLE}{generated/next-content.tex}
%
% HOW IT WORKS:
%   - Same auto-height measurement as leftbox/rightbox.
%   - Box width is always \FullBoxWidth (= \GridCols = full page).
%   - Has its own Y cursor (\FullPosY) so it doesn't interfere with
%     the left/right column cursors.
%   - After each box, Y auto-advances for the next box.
%
% STRUCTURAL CONSTANTS (derived from box-drawing pattern, not params):
%   7 = border/dot/space char count in title row
%   2 = left + right border columns (│ and ║)
%
% ============================================================================

% ----------------------------------------------------------------------------
% POSITION TRACKING (internal — set via \FullBoxInit)
% ----------------------------------------------------------------------------
\newcommand{\FullPosX}{0}               % Current X position
\newcommand{\FullPosY}{0}               % Current Y position

% Initialize position — CALL THIS FIRST in canvas.tex
\newcommand{\FullBoxInit}[2]{%
    % #1 = X position (grid columns from left, typically 0)
    % #2 = Y position (grid rows from top, typically \ContentStartY)
    \xdef\FullPosX{#1}%
    \xdef\FullPosY{#2}%
}

% ----------------------------------------------------------------------------
% CONTENT MEASUREMENT BOX (internal)
% ----------------------------------------------------------------------------
\newsavebox{\FullMeasureBox}

% ----------------------------------------------------------------------------
% MAIN BOX COMMAND
% ----------------------------------------------------------------------------
% Usage: \FullBox{TITLE}{content-file.tex}
%
\newcommand{\FullBox}[2]{%
    % #1 = TITLE (appears in top border)
    % #2 = content file path (e.g., generated/skills-table.tex)
    %
    % =========================================================================
    % STEP 1: MEASURE CONTENT HEIGHT
    % =========================================================================
    \renewcommand{\CurrentTypography}{\FullTypography}%
    \savebox{\FullMeasureBox}{%
        \begin{minipage}{\FullContentWidth\TPHorizModule}%
            \FullTypography%
            \input{#2}%
        \end{minipage}%
    }%
    %
    % Calculate rows needed
    \pgfmathsetmacro{\rawHeightPt}{\ht\FullMeasureBox + \dp\FullMeasureBox}%
    \pgfmathtruncatemacro{\contentRows}{ceil(\rawHeightPt / \TPVertModule)}%
    %
    % Total box height = top border + padding + content + padding + bottom border
    \pgfmathtruncatemacro{\boxRows}{1 + \FullBoxPadTop + \contentRows + \FullBoxPadBot + 1}%
    \pgfmathtruncatemacro{\bodyRows}{\boxRows - 2}%
    \pgfmathtruncatemacro{\innerW}{\FullBoxWidth - 2}%
    %
    % Debug output (visible in .log file)
    \typeout{FULLBOX [\detokenize{#1}]: rawPt=\rawHeightPt, contentRows=\contentRows, boxRows=\boxRows}%
    \LogBoxHeight{#2}{\contentRows}%
    %
    % =========================================================================
    % STEP 2: DRAW BOX FRAME
    % =========================================================================
    \StrLen{#1}[\titleLen]%
    \pgfmathtruncatemacro{\dashCount}{\FullBoxWidth - 7 - \titleLen}%
    %
    \begin{textblock}{\FullBoxWidth}(\FullPosX,\FullPosY)%
        \mono%
        % Kill all automatic spacing — critical for deterministic heights
        \baselineskip=0pt\relax%
        \lineskip=0pt\relax%
        \lineskiplimit=0pt\relax%
        \parskip=0pt\relax%
        \offinterlineskip%
        %
        % Top border with title
        \vbox to \TPVertModule{\vss\hbox{{\FSL┌─·} \textbf{#1} {\FSL·\Repeat{\dashCount}{─}}{\FDL╖}}\vss}%
        % Body rows (dot-filled)
        \foreach \n in {1,...,\bodyRows}{%
            \vbox to \TPVertModule{\vss\hbox{{\FSL│}{\FDOT\Repeat{\innerW}{·}}{\FDL║}}\vss}%
        }%
        % Bottom border
        \vbox to \TPVertModule{\vss\hbox{{\FDL╘\Repeat{\innerW}{═}╝}}\vss}%
    \end{textblock}%
    %
    % =========================================================================
    % STEP 3: PLACE CONTENT (two layers: background overlay + text)
    % =========================================================================
    %
    % Layer 1: Background rectangle covering dots inside the border,
    %          inset by 1 grid cell on each side so 1 dot column/row stays visible.
    \pgfmathsetmacro{\bgX}{\FullPosX + 2}%
    \pgfmathsetmacro{\bgY}{\FullPosY + 2}%
    \pgfmathtruncatemacro{\bgW}{\innerW - 3}%
    \pgfmathtruncatemacro{\bgH}{\bodyRows - 3}%
    \begin{textblock}{\bgW}(\bgX,\bgY)%
        \colorbox{content-bg}{\makebox[\bgW\TPHorizModule][l]{\rule{0pt}{\bgH\TPVertModule}}}%
    \end{textblock}%
    %
    % Layer 2: Content text, positioned with padding offsets on top of the bg.
    \pgfmathsetmacro{\contentX}{\FullPosX + \FullBoxPadLeft}%
    \pgfmathsetmacro{\contentY}{\FullPosY + 1 + \FullBoxPadTop}%
    \begin{textblock}{\FullContentWidth}(\contentX,\contentY)%
        \begin{minipage}{\FullContentWidth\TPHorizModule}%
            \FullTypography%
            \input{#2}%
        \end{minipage}%
    \end{textblock}%
    %
    % =========================================================================
    % STEP 4: UPDATE Y FOR NEXT BOX
    % =========================================================================
    \pgfmathsetmacro{\newY}{\FullPosY + \boxRows}%
    \xdef\FullPosY{\newY}%
}

% ----------------------------------------------------------------------------
% GAP COMMAND
% ----------------------------------------------------------------------------
% Usage: \FullBoxGap{1}     % Add 1 row gap
%        \FullBoxGap{0.5}   % Add half row gap
%
\newcommand{\FullBoxGap}[1]{%
    \pgfmathsetmacro{\newY}{\FullPosY + #1}%
    \xdef\FullPosY{\newY}%
}

% ----------------------------------------------------------------------------
% DEBUG: Show current Y position
% ----------------------------------------------------------------------------
% Usage: \FullBoxDebug      % Shows red "Y=xxx" at current position
%
\newcommand{\FullBoxDebug}{%
    \begin{textblock}{20}(\FullPosX,\FullPosY)%
        \tiny\mono\color{red}Y=\FullPosY%
    \end{textblock}%
}

\endinput
